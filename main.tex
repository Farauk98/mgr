%%%%%%%%%%%%%%%%%%%%%%%%%%%%%%%%%%%%%%%%%%%%%%%%
%%%%%%%%%%%%%Przykładowy dokument%%%%%%%%%%%%%%%
%%%%%%%%%%wraz z klasą pracadyp.cls%%%%%%%%%%%%%
%%%%%%%%%%%%%%%%%%%%%%%%%%%%%%%%%%%%%%%%%%%%%%%%

% w nawiasie kwadratowym wpisujemy rodzaj pracy: 
% magisterska, licencjacka, inzynierska
\documentclass[magisterska]{pracadypl}

%% ważne definicje %%
\usepackage{tgtermes}
\usepackage[T1]{fontenc}
\usepackage{polski}
\usepackage[utf8]{inputenc}
\input glyphtounicode
\pdfgentounicode=1
\usepackage{amssymb}
\usepackage{amsmath}
\usepackage{graphicx}
\usepackage{titlesec}
\usepackage{color}
\usepackage{xcolor}
\bibliographystyle{plain}

\def\mgr{magisterska}
\def\lic{licencjacka}
\def\inz{inżynierska}

\def\sk{Słowa kluczowe}
\def\kw{Keywords}
\def\et{Title in English}
%% koniec ważnych definicji %%



%% wypełnia Autor pracy %%

%autor pracy
\author{Rafał Kornat}
%numer albumu
\nralbumu{412521}
%tytuł pracy
\title{Badanie porównawcze strategii ofensywnych w grze "Statki" w kontekście efektywności i skuteczności wybranych metod}
%kierunek studiów
\kierunek{Informatyka}
%promotor w dopełniaczu
\opiekun{Dr-a Artura Lipnickiego}
\katedra{Katedra analizy Nieliniowej}
%rok
\date{2024}
%Słowa kluczowe:
\slkluczowe{pierwsze, drugie, trzecie, czwarte}
%tytuł po angielsku
\tytulang{Title in English}
%słowa kluczowe po angielsku
\keywords{first, second, third, fourth}
%% koniec ważnych definicji %%

%% APD %%
%% w systemie APD należy jeszcze wpisać, poza powyższymi informacjami, streszczenie oraz streszczenie w języku angielskim  %%


%%% definicje %%%
\def\pd{\noindent \textbf{Dowód.~}} %%początek dowodu
\def\kd{\hfill\mbox{$\rule{2mm}{2mm}$}} %%koniec dowodu
\newtheorem{defi}{Definicja}[section]
\newtheorem{uwaga}{Uwaga}[section]
\newtheorem{tw}{Twierdzenie}[section]
\newtheorem{lem}{Lemat}[section]
\newtheorem{wn}{Wniosek}[section]
\renewcommand\thetw{\thesection.\arabic{tw}.}
\renewcommand\thedefi{\thesection.\arabic{defi}.}
\renewcommand\theuwaga{\thesection.\arabic{uwaga}.}
\renewcommand\thetw{\thesection.\arabic{tw}.}
\renewcommand\thelem{\thesection.\arabic{lem}.}
\renewcommand\thewn{\thesection.\arabic{wn}.}
%
\definecolor{wmiigreen}{rgb}{0.0, 0.5, 0.0}
\titleformat{\chapter}[display]
  {\normalfont\huge\bfseries\color{wmiigreen}}{\chaptertitlename\ \thechapter}{10pt}{\Huge}
 %
\linespread{1.3}
%%% koniec definicji %%%


\begin{document}

\maketitle
\tableofcontents
\newpage



\chapter{Wstęp}

We wstępie pracy dyplomowej powinien znaleźć się opis wkładu własnego studenta w uzyskanie przedstawianych wyników a także informacje o podstawowych źródłach, na podstawie których student przygotował pracę.


\chapter{Podstawowe pojęcia}

\section{Definicje i własności}

W niniejszej części pracy podane zostaną pojęcia niezbędne w późniejszych rozważaniach (patrz \cite{Kostrykin} lub \cite{Lang}).
\begin{defi}
Niech $G$ będzie niepustym zbiorem. Działaniem w $G$ nazywamy dowolne odwzorowanie $\circ:G\times G\to G$.
\end{defi}

\begin{defi}
Niech $G$ będzie niepustym zbiorem, $\circ$ działaniem w $G$. Element $e\in G$ nazywamy neutralnym (działania $\circ$), jeśli dla każdego $a\in G$ mamy $a\circ e=e\circ a=a$.
\end{defi}

\begin{lem}\label{lem:element_neutralny}
Jeśli działanie $\circ$ w $G$ posiada element neutralny, to jest on jeden.
\end{lem}
\pd Niech $e,e'\in G$ będą dwoma elementami neutralnymi. Wtedy
\begin{equation}\label{eq:element_neutralny}
e=e'\circ e=e'.
\end{equation}
Zatem element neutralny jest jeden. \kd


\section{Przykłady}

Działaniem w zbiorze liczb naturalnych jest dodawanie, natomiast działaniem w tym zbiorze nie jest odejmowanie.




\chapter{Część główna}




\chapter{Rozdział badawczy}

Praca powinna spełniać wymogi formalne, merytoryczne i redakcyjne opisane w Regulaminie  Studiów (Rozdział IX) oraz w uchwale nr 184 Rady Wydziału Matematyki i Informatyki UŁ z dnia 25.09.2019 ze szczególnym uwzględnieniem wymogu, aby była ona samodzielnym opracowaniem zagadnienia naukowego lub praktycznego albo dokonaniem technicznym, prezentującym ogólną wiedzę i umiejętności studenta, związanym ze studiami na danym kierunku, poziomie i profilu oraz umiejętności samodzielnego analizowania i wnioskowania (Ustawa 2.0 Art. 76 p. 2)

Praca dyplomowa będąca pracą inżynierską powinna zawierać samodzielne opracowanie praktycznego problemu i może mieć charakter projektu, studium porównawczego lub opracowania analitycznego.




\chapter{Zakończenie}




\begin{thebibliography}{7}
\addcontentsline{toc}{chapter}{Bibliografia}
%
\bibitem{Lang}
Serge Lang, 
\textit{Algebra. Revised third edition}, 
New York, Springer-Verlag, 2002.
%
\bibitem{Kostrykin} 
Aleksiej Kostrykin, 
\textit{Wstęp do algebry. Podstawy algebry},
Warszawa, Wydawnictwo Naukowe PWN, 2022.
\end{thebibliography}
\end{document}