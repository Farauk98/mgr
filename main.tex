%%%%%%%%%%%%%%%%%%%%%%%%%%%%%%%%%%%%%%%%%%%%%%%%
%%%%%%%%%%%%%Przykładowy dokument%%%%%%%%%%%%%%%
%%%%%%%%%%wraz z klasą pracadyp.cls%%%%%%%%%%%%%
%%%%%%%%%%%%%%%%%%%%%%%%%%%%%%%%%%%%%%%%%%%%%%%%

% w nawiasie kwadratowym wpisujemy rodzaj pracy: 
% magisterska, licencjacka, inzynierska
\documentclass[magisterska]{pracadypl}

%% ważne definicje %%
\usepackage{tgtermes}
\usepackage[T1]{fontenc}
\usepackage{polski}
\usepackage[utf8]{inputenc}
\input glyphtounicode
\pdfgentounicode=1
\usepackage{amssymb}
\usepackage{amsmath}
\usepackage{graphicx}
\usepackage{titlesec}
\usepackage{color}
\usepackage{xcolor}
\bibliographystyle{plain}

\def\mgr{magisterska}
\def\lic{licencjacka}
\def\inz{inżynierska}

\def\sk{Słowa kluczowe}
\def\kw{Keywords}
\def\et{Title in English}
%% koniec ważnych definicji %%



%% wypełnia Autor pracy %%

%autor pracy
\author{Rafał Kornat}
%numer albumu
\nralbumu{412521}
%tytuł pracy
\title{Badanie porównawcze strategii ofensywnych w grze "Statki" w kontekście efektywności i skuteczności wybranych metod}
%kierunek studiów
\kierunek{Informatyka}
%promotor w dopełniaczu
\opiekun{Dr-a Artura Lipnickiego}
\katedra{Katedra analizy Nieliniowej}
%rok
\date{2024}
%Słowa kluczowe:
\slkluczowe{pierwsze, drugie, trzecie, czwarte}
%tytuł po angielsku
\tytulang{Title in English}
%słowa kluczowe po angielsku
\keywords{first, second, third, fourth}
%% koniec ważnych definicji %%

%% APD %%
%% w systemie APD należy jeszcze wpisać, poza powyższymi informacjami, streszczenie oraz streszczenie w języku angielskim  %%


%%% definicje %%%
\def\pd{\noindent \textbf{Dowód.~}} %%początek dowodu
\def\kd{\hfill\mbox{$\rule{2mm}{2mm}$}} %%koniec dowodu
\newtheorem{defi}{Definicja}[section]
\newtheorem{uwaga}{Uwaga}[section]
\newtheorem{tw}{Twierdzenie}[section]
\newtheorem{lem}{Lemat}[section]
\newtheorem{wn}{Wniosek}[section]
\renewcommand\thetw{\thesection.\arabic{tw}.}
\renewcommand\thedefi{\thesection.\arabic{defi}.}
\renewcommand\theuwaga{\thesection.\arabic{uwaga}.}
\renewcommand\thetw{\thesection.\arabic{tw}.}
\renewcommand\thelem{\thesection.\arabic{lem}.}
\renewcommand\thewn{\thesection.\arabic{wn}.}
%
\definecolor{wmiigreen}{rgb}{0.0, 0.5, 0.0}
\titleformat{\chapter}[display]
  {\normalfont\huge\bfseries\color{wmiigreen}}{\chaptertitlename\ \thechapter}{10pt}{\Huge}
 %
\linespread{1.3}
%%% koniec definicji %%%


\begin{document}

\maketitle
\tableofcontents
\newpage


\chapter{Wstęp}

\section{Zasady gry}
Klasyczna gra w "Statki" to strategiczna rozgrywka dla dwóch osób, której 
celem jest zatopienie wszystkich okrętów przeciwnika. 
Każdy gracz posiada dwie plansze: jedną do rozmieszczenia swoich statków, a drugą do zaznaczania strzałów oddanych w stronę rywala. 
Plansze są rozmiaru 10x10, są one oznaczone odpowiednio literami od A do J w poziomie i cyframi od 1 do 10 w pionie.
Flota każdego z graczy składa się: 
\begin{itemize}
  \item jednego lotniskowca (pięć pól), 
  \item jednego pancernika (cztery pola), 
  \item jednego krążownika (trzy pola),
  \item jednego okrętu podwodnego (trzy pola),
  \item jednego niszczyciela (dwa pola).
\end{itemize}
Statki rozmieszczane są na planszy w pozycji pionowej lub poziomej i do końca gry nie mogą zmieniać swojej lokalizacji. 
Okręty mogą stykać się bokami lub rogami, co stanowi odstępstwo od klasycznych zasad, gdzie takie zachowanie jest zabronione.
Rozgrywka odbywa się w turach, w których gracze wykonują strzały na przemian.
W celu oddania strzału, gracz podaje współrzędne pola, na przykład B5. 
Następnie przeciwnik sprawdza i informuje, czy na podanym polu znajduje się statek.
Mówi słowo "pudło" w przypadku, gdy na danym polu nie ma statku, a "trafiony" w przeciwnym przypadku.
Gdy wszystkie pola danego statku są trafione, statek zostaje zatopiony, a właściciel statku informuje oponenta o jego zatopieniu.
Gra kończy się, gdy jedna z osób jako pierwsza zatopi wszystkie okręty wroga.
\chapter{Wstęp}
\section{Podstawowe Pojecia}

\subsection{Definicje oraz Twierdzenia - Statystyka}

% Definicja wartości oczekiwanej dla zmiennej dyskretnej
W trakcie analizy strategii w grze "Statki" będziemy operować skończonymi przestrzeniami.
Z tego powodu będziemy korzystać z ustalonego nazewnictwa, które zostanie zaczerpnięte z książki \cite{Statystyka_1} 

\begin{defi}[Wartośc oczekiwana dla zmiennej dyskretnej]\cite{Statystyka_1} 

Jeśli \( X \) jest dyskretną zmienną losową przyjmującą wartości \( x_1, x_2, \ldots \) z prawdopodobieństwami \( p_1, p_2, \ldots \), to wartość oczekiwana \( \mathbb{E}(X) \) jest dana wzorem:
\[
\mathbb{E}(X) = \sum_{i=1}^{\infty} x_i \cdot p_i
\]
\end{defi}
\begin{defi}[Wariancja dla zmiennej dyskretnej]\cite{Statystyka_1} 

Jeśli \( X \) jest dyskretną zmienną losową przyjmującą wartości \( x_1, x_2, \ldots \) z prawdopodobieństwami \( p_1, p_2, \ldots \) oraz wartością oczekiwaną \( \mathbb{E}(X) \), to wariancja \( \mathrm{Var}(X) \) jest dana wzorem:
\[
\mathrm{Var}(X) = \mathbb{E}\left[(X - \mathbb{E}(X))^2\right] = \sum_{i=1}^{\infty} (x_i - \mathbb{E}(X))^2 \cdot p_i
\]
\end{defi}
\begin{defi}[Odchylenie standardowe dla zmiennej dyskretnej]\cite{Statystyka_1} 

Jeśli \( X \) jest dyskretną zmienną losową, to odchylenie standardowe \( \sigma \) jest dane wzorem:
\[
\sigma = \sqrt{\mathrm{Var}(X)}
\]
\end{defi}

Test chi-kwadrat jest testem statystycznym używanym do oceny, czy obserwowane częstości zdarzeń różnią się istotnie od tych spodziewanych w założonym modelu. Jest powszechnie stosowany w analizie danych kategorycznych.

\begin{defi}[Test chi-kwadrat]
  Wartość statystyki chi-kwadrat (\( \chi^2 \)) dla testu chi-kwadrat jest obliczana na podstawie porównania między obserwowanymi (\( O_i \)) a spodziewanymi (\( E_i \)) częstościami w poszczególnych kategoriach. Dla \( k \) kategorii, wzór na \( \chi^2 \) jest następujący:
  \[ \chi^2 = \sum_{i=1}^{k} \frac{(O_i - E_i)^2}{E_i} \]
  
  W przypadku dużych prób, statystyka ta ma rozkład chi-kwadrat z \( k-1 \) stopniami swobody.
\end{defi}

\begin{thebibliography}{7}
\addcontentsline{toc}{chapter}{Bibliografia}
\bibitem{Statystyka_1} 
Jacek Koronacki, Jan Mielniczuk 
\textit{Statystyka dla studentów kierunków technicznych i przyrodniczych},
Warszawa, Wydawnictwo Naukowo-Techniczne.
\end{thebibliography}
\end{document}